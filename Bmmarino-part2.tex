
\renewcommand{\outlinei}{enumerate}
\renewcommand{\outlineii}{itemize}
\begin{outline}
    \1 \textbf{Introductory context}
        \2 Mechanical instabilities enable many of the responsive behaviors seen in nature, such as Venus fly traps \cite{Baumgartner_VenusFlyTrap}. Recent work has identified the usefulness of these mechanical instabilities in soft robotic systems \cite{Pal_ExploitingInstab}, enabling localized responsive behaviors and intelligent functionality. These works have focused on instabilities ranging from multistable shells, buckling and more \cite{Lu_HarnessingInstab}. Soft beam buckling has been identified as one of these key instabilities, and begun to be designed with more complex mechanical behaviors achieve broader functionalities \cite{Narayanan_LeveragingInstab}.
    \1 \textbf{The state of the field}
        \2 Vertical beam buckling has been primarily studied for slender beams made out of elastic materials \cite{TimoshenkoElastStab}. Recent work has begun to push the boundaries vertical beam buckling, extending modeling to both hyperelastic beams, and to beams that can no longer be considered slender \cite{Chen&Jin_Buckling}, \cite{Chen&Jin_PostBuckling}.
        \2 Viscoelasticity however, is not an untouched area in beam buckling \cite{Spanguolo_LargePlanarDef}. Visco-elastic effects have been considered and derived in slender elastic beams \cite{Hilton_TimoViscoBeam} \cite{Vinogradov_ViscoBeamColumns}. For hyperelastic structures, such as plates, slender beams and frames, quasi-linear viscoelasticity has been implemented, primarily via FEA and semi-analytical methods \cite{Dadgar_2dViscoHyperBeamFrames} \cite{Dastjerdi_ViscoLargeDefPlate}.
    \1 \textbf{The Big Gap}
        \2 There has not been analysis or study of non-linear visco-elastic behaviors in wide, hyperelastic beams, due to the recentness of interest in such behaviors, lack of generalized modeling and lack of data for validation.
\end{outline}