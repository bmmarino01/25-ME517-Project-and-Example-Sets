
\section*{Project I: Topic ID and Overview (due Sept 19)}
The spontaneous collapse of vertical beams, known as buckling, is a structural phenomena that has been observed and studied in rigid structures since the 1750s. Recent research efforts have begun studying this phenomena in a range of structures and materials, moving from purely rigid metal structures, to composites and  'soft' structures. 

Recent work on soft robotic systems have used this buckling behavior in soft materials, where the collapse of the structure remains in the elastic deformation region of soft materials, enabling repeatable, controlled buckling of a soft structure. This can be used to modulate energy absorption, achieve 'intelligent' responses and control mechanical behaviors.

Beam buckling 

\begin{enumerate}
\item \textbf{Statement of Research Interest (why you personally want to study the subject)}
\begin{itemize}
\item Soft beam buckling 
\item the usage of buckling behaviors in soft structures to enable complex, intelligent responses based on the local structural and material properties of a system
\end{itemize}

\item \textbf{Intellectual Merit (why it is objectively worth delving deeper)}

\item \textbf{Broader Impact (who, or what, does studying this area benefit?)}
\begin{itemize}
    \item advances in our understanding of of soft beam buckling improves 
    \item understandings of viscoelastic behaviors in soft buckling beams could cascade to industries such as robotics, enabling skin-like behavior in soft surfaces, and other localized intelligent responses in soft robots
\end{itemize}

\end{enumerate}



