
\section*{Project I: Topic ID and Overview (due Sept 19)}
The spontaneous collapse of vertical beams, known as buckling, is a structural phenomena that has been observed and studied in rigid structures since the 1750s \cite{TimoshenkoElastStab}. Recent research efforts have moved to studying this phenomena in a range of structures and materials, moving from purely rigid metal structures, to composites and 'soft' structures. How 'soft' these structures are has steadily increased with time until the present, where buckling of completely hyperelastic beams has been studied. Recent work on soft robotic systems have used this buckling behavior in soft structures, where the collapse of the structure remains in the elastic deformation regime of the soft material, enabling repeatable, controlled buckling of a soft structure. This can be used to modulate energy absorption, achieve 'intelligent' responses and control mechanical behaviors \cite{Pal_ExploitingInstab}. The ability to control and modulate behaviors like this will be used to enhance the functionally of soft robots and other systems that utilize soft materials.\\

In recent publications, analytical solutions to vertical beam buckling based on the Neo-Hookean hyperelastic model have been derived \cite{Chen&Jin_Buckling} \cite{Chen&Jin_PostBuckling}. Despite these efforts, no further work has been done to identify the often overlooked viscoelastic behaviors of these soft vertical beams, and how they affect values such as the critical buckling load. These viscoelastic behaviors present an underutilized space to modify the performance of these beams, and better harness the natural properties of the soft materials for further useful behaviors. Given the expected topics in this class, material modeling for viscoelastic behaviors should provide the resources needed to begin adapting the already existing vertical beam buckling hyperelastic models to include viscoelastic behavior. Supporting mathematical knowledge about how to solve the partial differential equations and signal processing that will be produced will be key in generating analytical solutions for the viscoelastic vertical beams.\\

Proper understanding of viscoelastic buckling behaviors in purely soft vertical beams has implications for other buckling behaviors in soft materials. An example of this is the buckling of sheets, spheres and other surfaces, which are common shapes for objects like balls, balloons, packaging, tires, hoses, gloves and many more. Improvements in the understanding of viscoelastic behaviors of these buckling modes could help reduce breakage and failures, while better achieving the desired outcomes by harnessing these behaviors. More specifically in the field of soft robotics, understanding viscoelastic beam buckling and other soft material structural phenomena work towards the development of localized responses in soft robotics systems, such as soft robotic skin being able to emulate lifelike behaviors. The ability to emulate these behaviors in soft robotic systems will benefit the functionality of robots that interact with humans, and the safety of these systems. \\
